\documentclass[12pt]{article}
\title{How to cite references in Latex}
\author{Prasad Patil}
\usepackage[citestyle=authoryear-icomp,bibstyle=nature,ibidtracker=false,sorting =ynt]{biblatex}

\addbibresource{references.bib}
\begin{document}
\maketitle
\hrule
\hspace{2cm}

This is an article to show how to cite references first i will start some other article  

\begin{abstract}
  This is Practical on Bibliography where I hava used all the parts that we have learnt in previous practicles .
\end{abstract}


\section{Introduction}
This document is a model and instructions for \LaTeX.
Please observe the conference page limits.

\section{Ease of Use}

\subsection{Maintaining the Integrity of the Specifications}

The IEEEtran class file is used to format your paper and style the text. All margins,
column widths, line spaces, and text fonts are prescribed; please do not
alter them. You may note peculiarities. For example, the head margin
measures proportionately more than is customary. This measurement
and others are deliberate, using specifications that anticipate your paper
as one part of the entire proceedings, and not as an independent document.
Please do not revise any of the current designations.

\section{Prepare Your Paper Before Styling}
Before you begin to format your paper, first write and save the content as a
separate text file. Complete all content and organizational editing before
formatting. Please note sections \ref{AA}--\ref{SCM} below for more information on
proofreading, spelling and grammar.

Keep your text and graphic files separate until after the text has been
formatted and styled. Do not number text heads---{\LaTeX} will do that
for you.

\subsection{Abbreviations and Acronyms}\label{AA}
Define abbreviations and acronyms the first time they are used in the text,
even after they have been defined in the abstract. Abbreviations such as
IEEE, SI, MKS, CGS, ac, dc, and rms do not have to be defined. Do not use
abbreviations in the title or heads unless they are unavoidable.

\subsection{Units}
\begin{itemize}
  \item Use either SI (MKS) or CGS as primary units. (SI units are encouraged.) English units may be used as secondary units (in parentheses). An exception would be the use of English units as identifiers in trade, such as ``3.5-inch disk drive''.
  \item Avoid combining SI and CGS units, such as current in amperes and magnetic field in oersteds. This often leads to confusion because equations do not balance dimensionally. If you must use mixed units, clearly state the units for each quantity that you use in an equation.
  \item Do not mix complete spellings and abbreviations of units: ``Wb/m\textsuperscript{2}'' or ``webers per square meter'', not ``webers/m\textsuperscript{2}''. Spell out units when they appear in text: ``. . . a few henries'', not ``. . . a few H''.
  \item Use a zero before decimal points: ``0.25'', not ``.25''. Use ``cm\textsuperscript{3}'', not ``cc''.)
\end{itemize}

\subsection{Equations}
Number equations consecutively. To make your
equations more compact, you may use the solidus (~/~), the exp function, or
appropriate exponents. Italicize Roman symbols for quantities and variables,
but not Greek symbols. Use a long dash rather than a hyphen for a minus
sign. Punctuate equations with commas or periods when they are part of a
sentence, as in:
\begin{equation}
  a+b=\gamma\label{eq}
\end{equation}

Be sure that the
symbols in your equation have been defined before or immediately following


\subsection{\LaTeX-Specific Advice}


to combine sections, add equations, or change the order of figures or
citations without having to go through the file line by line.

Please don't use the \verb|{eqnarray}| equation environment. Use
\verb|{align}| or \verb|{IEEEeqnarray}| instead. The \verb|{eqnarray}|
environment leaves unsightly spaces around relation symbols.

Please note that the \verb|{subequations}| environment in {\LaTeX}
will increment the main equation counter even when there are no
equation numbers displayed. If you forget that, you might write an
article in which the equation numbers skip from (17) to (20), causing
the copy editors to wonder if you've discovered a new method of
counting.
\printbibliography[heading=bibintoc]
\begin{thebibliography} {}

\bibitem {aa}  title={A common directional tuning mechanism of Drosophila motion-sensing neurons in the ON and in the OFF pathway},
  author={Haag, Juergen and Mishra, Abhishek and Borst, Alexander},
  journal={Elife},
  volume={6},
  pages={e29044},
  year={2017},
  publisher={eLife Sciences Publications Limited}

\bibitem{trishna} Trishna Ugale.,Rucha Kardile.,Stock Price Predictions using Crossover SMA,978-1-6654-1703-7/21.

\bibitem{pqr} PQR,Latex ,IEEE

\bibitem{sachi} Sachi Nandan Mohanty.,Rucha Kardile.,Stock Price Predictions using Crossover SMA,978-1-6654-1703-7/21.

\bibitem{rucha} Trishna Ugale.,Sachi Nandan Mohanty.,Stock Price Predictions using Crossover SMA,978-1-6654-1703-7/21.

\bibitem{abc} ABC,Research Paper,2022,IEEE.

\bibitem Source types:
article, book, incollection (for book chapter), inproceedings, techreport, unpublished, Patent, misc (online, presentation, video, etc.)


\bibitem Article:
Required fields: author, title, journal, year
Optional fields: volume, number, pages, month, note

\bibitem Book:
Required fields: author/editor, title, publisher, year
Optional fields: volume/number, series, address, edition, month, note

\bibitem Incollection:
Required fields: author, title, booktitle, publisher, year
Optional fields: editor, volume/number, series, type, chapter, pages, address, edition, month, note

\bibitem Proceedings: 
Required fields: title, year
Optional fields: editor, volume/number, series, address, month, organization, publisher, note

\bibitem Misc: 
All fields are optional

\bibitem Techreport 
Required fields: author, title, institution, year
Optional fields: type, number, address, month, note

\bibitem Unpublished 
Required fields: author, title, note
Optional fields: month, year

\end{thebibliography} 
\end{document}